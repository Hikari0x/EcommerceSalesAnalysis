%% Generated by Sphinx.
\def\sphinxdocclass{report}
\documentclass[letterpaper,10pt,english]{sphinxmanual}
\ifdefined\pdfpxdimen
   \let\sphinxpxdimen\pdfpxdimen\else\newdimen\sphinxpxdimen
\fi \sphinxpxdimen=.75bp\relax
\ifdefined\pdfimageresolution
    \pdfimageresolution= \numexpr \dimexpr1in\relax/\sphinxpxdimen\relax
\fi
\newdimen\sphinxremdimen\sphinxremdimen = 10pt
%% let collapsible pdf bookmarks panel have high depth per default
\PassOptionsToPackage{bookmarksdepth=5}{hyperref}
%% turn off hyperref patch of \index as sphinx.xdy xindy module takes care of
%% suitable \hyperpage mark-up, working around hyperref-xindy incompatibility
\PassOptionsToPackage{hyperindex=false}{hyperref}
%% memoir class requires extra handling
\makeatletter\@ifclassloaded{memoir}
{\ifdefined\memhyperindexfalse\memhyperindexfalse\fi}{}\makeatother

\PassOptionsToPackage{booktabs}{sphinx}
\PassOptionsToPackage{colorrows}{sphinx}

\PassOptionsToPackage{warn}{textcomp}

\catcode`^^^^00a0\active\protected\def^^^^00a0{\leavevmode\nobreak\ }
\usepackage{cmap}
\usepackage{xeCJK}
\usepackage{amsmath,amssymb,amstext}
\usepackage{babel}



\setmainfont{FreeSerif}[
  Extension      = .otf,
  UprightFont    = *,
  ItalicFont     = *Italic,
  BoldFont       = *Bold,
  BoldItalicFont = *BoldItalic
]
\setsansfont{FreeSans}[
  Extension      = .otf,
  UprightFont    = *,
  ItalicFont     = *Oblique,
  BoldFont       = *Bold,
  BoldItalicFont = *BoldOblique,
]
\setmonofont{FreeMono}[Scale=0.9,
  Extension      = .otf,
  UprightFont    = *,
  ItalicFont     = *Oblique,
  BoldFont       = *Bold,
  BoldItalicFont = *BoldOblique,
]



\usepackage[Sonny]{fncychap}
\ChNameVar{\Large\normalfont\sffamily}
\ChTitleVar{\Large\normalfont\sffamily}
\usepackage{sphinx}

\fvset{fontsize=\small,formatcom=\xeCJKVerbAddon}
\usepackage{geometry}


% Include hyperref last.
\usepackage{hyperref}
% Fix anchor placement for figures with captions.
\usepackage{hypcap}% it must be loaded after hyperref.
% Set up styles of URL: it should be placed after hyperref.
\urlstyle{same}

\addto\captionsenglish{\renewcommand{\contentsname}{内容目录:}}

\usepackage{sphinxmessages}
\setcounter{tocdepth}{1}



\title{API参考}
\date{2025 年 12 月 27 日}
\release{v1.0}
\author{lihetong}
\newcommand{\sphinxlogo}{\vbox{}}
\renewcommand{\releasename}{发行版本}
\makeindex
\begin{document}

\ifdefined\shorthandoff
  \ifnum\catcode`\=\string=\active\shorthandoff{=}\fi
  \ifnum\catcode`\"=\active\shorthandoff{"}\fi
\fi

\pagestyle{empty}
\sphinxmaketitle
\pagestyle{plain}
\sphinxtableofcontents
\pagestyle{normal}
\phantomsection\label{\detokenize{index::doc}}


\sphinxAtStartPar
欢迎来到 EcommerceSalesAnalysis 项目文档!

\sphinxAtStartPar
这是一个电商平台销售数据分析项目,包含数据加载、清洗、探索、可视化、特征工程、模型训练和评估等模块。


\chapter{介绍}
\label{\detokenize{index:id1}}
\sphinxAtStartPar
这是一个电商平台销售数据分析系统,主要功能包括:
\begin{itemize}
\item {} 
\sphinxAtStartPar
数据加载与预处理

\item {} 
\sphinxAtStartPar
数据探索与分析

\item {} 
\sphinxAtStartPar
数据可视化

\item {} 
\sphinxAtStartPar
特征工程

\item {} 
\sphinxAtStartPar
模型训练与评估

\end{itemize}


\chapter{模块参考}
\label{\detokenize{index:id2}}

\begin{savenotes}
\sphinxatlongtablestart
\sphinxthistablewithglobalstyle
\sphinxthistablewithnovlinesstyle
\makeatletter
  \LTleft \@totalleftmargin plus1fill
  \LTright\dimexpr\columnwidth-\@totalleftmargin-\linewidth\relax plus1fill
\makeatother
\begin{longtable}{\X{1}{2}\X{1}{2}}
\sphinxtoprule
\endfirsthead

\multicolumn{2}{c}{\sphinxnorowcolor
    \makebox[0pt]{\sphinxtablecontinued{\tablename\ \thetable{} \textendash{} 接上页}}%
}\\
\sphinxtoprule
\endhead

\sphinxbottomrule
\multicolumn{2}{r}{\sphinxnorowcolor
    \makebox[0pt][r]{\sphinxtablecontinued{续下页}}%
}\\
\endfoot

\endlastfoot
\sphinxtableatstartofbodyhook
\begin{varwidth}[t]{\sphinxcolwidth{1}{2}}
\sphinxAtStartPar
{\hyperref[\detokenize{api/data_loader:module-data_loader}]{\sphinxcrossref{\sphinxcode{\sphinxupquote{data\_loader}}}}}
\sphinxbeforeendvarwidth
\end{varwidth}%
&\begin{varwidth}[t]{\sphinxcolwidth{1}{2}}
\sphinxAtStartPar

\sphinxbeforeendvarwidth
\end{varwidth}%
\\
\sphinxhline\begin{varwidth}[t]{\sphinxcolwidth{1}{2}}
\sphinxAtStartPar
{\hyperref[\detokenize{api/data_clean:module-data_clean}]{\sphinxcrossref{\sphinxcode{\sphinxupquote{data\_clean}}}}}
\sphinxbeforeendvarwidth
\end{varwidth}%
&\begin{varwidth}[t]{\sphinxcolwidth{1}{2}}
\sphinxAtStartPar

\sphinxbeforeendvarwidth
\end{varwidth}%
\\
\sphinxhline\begin{varwidth}[t]{\sphinxcolwidth{1}{2}}
\sphinxAtStartPar
{\hyperref[\detokenize{api/data_explore:module-data_explore}]{\sphinxcrossref{\sphinxcode{\sphinxupquote{data\_explore}}}}}
\sphinxbeforeendvarwidth
\end{varwidth}%
&\begin{varwidth}[t]{\sphinxcolwidth{1}{2}}
\sphinxAtStartPar

\sphinxbeforeendvarwidth
\end{varwidth}%
\\
\sphinxhline\begin{varwidth}[t]{\sphinxcolwidth{1}{2}}
\sphinxAtStartPar
{\hyperref[\detokenize{api/model_train:module-model_train}]{\sphinxcrossref{\sphinxcode{\sphinxupquote{model\_train}}}}}
\sphinxbeforeendvarwidth
\end{varwidth}%
&\begin{varwidth}[t]{\sphinxcolwidth{1}{2}}
\sphinxAtStartPar

\sphinxbeforeendvarwidth
\end{varwidth}%
\\
\sphinxhline\begin{varwidth}[t]{\sphinxcolwidth{1}{2}}
\sphinxAtStartPar
{\hyperref[\detokenize{api/model_evaluate:module-model_evaluate}]{\sphinxcrossref{\sphinxcode{\sphinxupquote{model\_evaluate}}}}}
\sphinxbeforeendvarwidth
\end{varwidth}%
&\begin{varwidth}[t]{\sphinxcolwidth{1}{2}}
\sphinxAtStartPar

\sphinxbeforeendvarwidth
\end{varwidth}%
\\
\sphinxhline\begin{varwidth}[t]{\sphinxcolwidth{1}{2}}
\sphinxAtStartPar
{\hyperref[\detokenize{api/config:module-config}]{\sphinxcrossref{\sphinxcode{\sphinxupquote{config}}}}}
\sphinxbeforeendvarwidth
\end{varwidth}%
&\begin{varwidth}[t]{\sphinxcolwidth{1}{2}}
\sphinxAtStartPar

\sphinxbeforeendvarwidth
\end{varwidth}%
\\
\sphinxbottomrule
\end{longtable}
\sphinxtableafterendhook
\sphinxatlongtableend
\end{savenotes}

\sphinxstepscope


\section{data\_loader}
\label{\detokenize{api/data_loader:module-data_loader}}\label{\detokenize{api/data_loader:data-loader}}\label{\detokenize{api/data_loader::doc}}\index{module@\spxentry{module}!data\_loader@\spxentry{data\_loader}}\index{data\_loader@\spxentry{data\_loader}!module@\spxentry{module}}\subsubsection*{Functions}


\begin{savenotes}
\sphinxatlongtablestart
\sphinxthistablewithglobalstyle
\sphinxthistablewithnovlinesstyle
\makeatletter
  \LTleft \@totalleftmargin plus1fill
  \LTright\dimexpr\columnwidth-\@totalleftmargin-\linewidth\relax plus1fill
\makeatother
\begin{longtable}{\X{1}{2}\X{1}{2}}
\sphinxtoprule
\endfirsthead

\multicolumn{2}{c}{\sphinxnorowcolor
    \makebox[0pt]{\sphinxtablecontinued{\tablename\ \thetable{} \textendash{} 接上页}}%
}\\
\sphinxtoprule
\endhead

\sphinxbottomrule
\multicolumn{2}{r}{\sphinxnorowcolor
    \makebox[0pt][r]{\sphinxtablecontinued{续下页}}%
}\\
\endfoot

\endlastfoot
\sphinxtableatstartofbodyhook
\begin{varwidth}[t]{\sphinxcolwidth{1}{2}}
\sphinxAtStartPar
{\hyperref[\detokenize{api/data_loader:data_loader.data_loader}]{\sphinxcrossref{\sphinxcode{\sphinxupquote{data\_loader}}}}}()
\sphinxbeforeendvarwidth
\end{varwidth}%
&\begin{varwidth}[t]{\sphinxcolwidth{1}{2}}
\sphinxAtStartPar
从数据库中获取数据或者是从csv中读取数据 获取数据基本结构 :return:返回原始数据
\sphinxbeforeendvarwidth
\end{varwidth}%
\\
\sphinxhline\begin{varwidth}[t]{\sphinxcolwidth{1}{2}}
\sphinxAtStartPar
{\hyperref[\detokenize{api/data_loader:data_loader.get_basic_info}]{\sphinxcrossref{\sphinxcode{\sphinxupquote{get\_basic\_info}}}}}()
\sphinxbeforeendvarwidth
\end{varwidth}%
&\begin{varwidth}[t]{\sphinxcolwidth{1}{2}}
\sphinxAtStartPar
获取数据基本信息
\sphinxbeforeendvarwidth
\end{varwidth}%
\\
\sphinxhline\begin{varwidth}[t]{\sphinxcolwidth{1}{2}}
\sphinxAtStartPar
{\hyperref[\detokenize{api/data_loader:data_loader.get_mysql_engine}]{\sphinxcrossref{\sphinxcode{\sphinxupquote{get\_mysql\_engine}}}}}()
\sphinxbeforeendvarwidth
\end{varwidth}%
&\begin{varwidth}[t]{\sphinxcolwidth{1}{2}}
\sphinxAtStartPar
获取数据库连接 :return:
\sphinxbeforeendvarwidth
\end{varwidth}%
\\
\sphinxhline\begin{varwidth}[t]{\sphinxcolwidth{1}{2}}
\sphinxAtStartPar
{\hyperref[\detokenize{api/data_loader:data_loader.load_from_mysql}]{\sphinxcrossref{\sphinxcode{\sphinxupquote{load\_from\_mysql}}}}}(engine, table\_name)
\sphinxbeforeendvarwidth
\end{varwidth}%
&\begin{varwidth}[t]{\sphinxcolwidth{1}{2}}
\sphinxAtStartPar
从 MySQL 读取整张表数据 :param engine: 数据库 :param table\_name: 表名 :return: DataFrame原始数据集
\sphinxbeforeendvarwidth
\end{varwidth}%
\\
\sphinxhline\begin{varwidth}[t]{\sphinxcolwidth{1}{2}}
\sphinxAtStartPar
{\hyperref[\detokenize{api/data_loader:data_loader.load_raw_data}]{\sphinxcrossref{\sphinxcode{\sphinxupquote{load\_raw\_data}}}}}()
\sphinxbeforeendvarwidth
\end{varwidth}%
&\begin{varwidth}[t]{\sphinxcolwidth{1}{2}}
\sphinxAtStartPar
加载原始用户数据,并规范列名 \sphinxhyphen{} 读取CSV / 数据库 \sphinxhyphen{} 不做任何清洗与修改 \sphinxhyphen{} 保证“原始性” :return:返回加载好的原始数据
\sphinxbeforeendvarwidth
\end{varwidth}%
\\
\sphinxhline\begin{varwidth}[t]{\sphinxcolwidth{1}{2}}
\sphinxAtStartPar
{\hyperref[\detokenize{api/data_loader:data_loader.save_to_mysql}]{\sphinxcrossref{\sphinxcode{\sphinxupquote{save\_to\_mysql}}}}}(df, table\_name, engine)
\sphinxbeforeendvarwidth
\end{varwidth}%
&\begin{varwidth}[t]{\sphinxcolwidth{1}{2}}
\sphinxAtStartPar
load\_raw\_data已经清洗过列名,去除空格,可以正常存入 先使用if下面的代码导入数据,然后使用if判断数据没问题 :param df:原始数据 :param table\_name:提前设定好的表名 :param engine:数据库
\sphinxbeforeendvarwidth
\end{varwidth}%
\\
\sphinxhline\begin{varwidth}[t]{\sphinxcolwidth{1}{2}}
\sphinxAtStartPar
{\hyperref[\detokenize{api/data_loader:data_loader.table_has_data}]{\sphinxcrossref{\sphinxcode{\sphinxupquote{table\_has\_data}}}}}(engine, table\_name)
\sphinxbeforeendvarwidth
\end{varwidth}%
&\begin{varwidth}[t]{\sphinxcolwidth{1}{2}}
\sphinxAtStartPar

\sphinxbeforeendvarwidth
\end{varwidth}%
\\
\sphinxbottomrule
\end{longtable}
\sphinxtableafterendhook
\sphinxatlongtableend
\end{savenotes}
\index{data\_loader()(在 data\_loader 模块中)@\spxentry{data\_loader()(在 data\_loader 模块中)}}

\begin{fulllineitems}
\phantomsection\label{\detokenize{api/data_loader:data_loader.data_loader}}
\pysigstartsignatures
\pysiglinewithargsret
{\sphinxcode{\sphinxupquote{data\_loader.}}\sphinxbfcode{\sphinxupquote{data\_loader}}}
{}
{{ $\rightarrow$ DataFrame}}
\pysigstopsignatures
\sphinxAtStartPar
从数据库中获取数据或者是从csv中读取数据
获取数据基本结构
:return:返回原始数据

\end{fulllineitems}

\index{get\_basic\_info()(在 data\_loader 模块中)@\spxentry{get\_basic\_info()(在 data\_loader 模块中)}}

\begin{fulllineitems}
\phantomsection\label{\detokenize{api/data_loader:data_loader.get_basic_info}}
\pysigstartsignatures
\pysiglinewithargsret
{\sphinxcode{\sphinxupquote{data\_loader.}}\sphinxbfcode{\sphinxupquote{get\_basic\_info}}}
{}
{{ $\rightarrow$ dict}}
\pysigstopsignatures\begin{quote}

\sphinxAtStartPar
获取数据基本信息
\end{quote}

\sphinxAtStartPar
:param df:传入原始数据
:return:返回部分数据

\end{fulllineitems}

\index{get\_mysql\_engine()(在 data\_loader 模块中)@\spxentry{get\_mysql\_engine()(在 data\_loader 模块中)}}

\begin{fulllineitems}
\phantomsection\label{\detokenize{api/data_loader:data_loader.get_mysql_engine}}
\pysigstartsignatures
\pysiglinewithargsret
{\sphinxcode{\sphinxupquote{data\_loader.}}\sphinxbfcode{\sphinxupquote{get\_mysql\_engine}}}
{}
{}
\pysigstopsignatures
\sphinxAtStartPar
获取数据库连接
:return:

\end{fulllineitems}

\index{load\_from\_mysql()(在 data\_loader 模块中)@\spxentry{load\_from\_mysql()(在 data\_loader 模块中)}}

\begin{fulllineitems}
\phantomsection\label{\detokenize{api/data_loader:data_loader.load_from_mysql}}
\pysigstartsignatures
\pysiglinewithargsret
{\sphinxcode{\sphinxupquote{data\_loader.}}\sphinxbfcode{\sphinxupquote{load\_from\_mysql}}}
{\sphinxparam{\DUrole{n}{engine}}\sphinxparamcomma \sphinxparam{\DUrole{n}{table\_name}\DUrole{p}{:}\DUrole{w}{ }\DUrole{n}{str}}}
{{ $\rightarrow$ DataFrame}}
\pysigstopsignatures
\sphinxAtStartPar
从 MySQL 读取整张表数据
:param engine: 数据库
:param table\_name: 表名
:return: DataFrame原始数据集

\end{fulllineitems}

\index{load\_raw\_data()(在 data\_loader 模块中)@\spxentry{load\_raw\_data()(在 data\_loader 模块中)}}

\begin{fulllineitems}
\phantomsection\label{\detokenize{api/data_loader:data_loader.load_raw_data}}
\pysigstartsignatures
\pysiglinewithargsret
{\sphinxcode{\sphinxupquote{data\_loader.}}\sphinxbfcode{\sphinxupquote{load\_raw\_data}}}
{}
{{ $\rightarrow$ DataFrame}}
\pysigstopsignatures
\sphinxAtStartPar
加载原始用户数据,并规范列名
\sphinxhyphen{} 读取CSV / 数据库
\sphinxhyphen{} 不做任何清洗与修改
\sphinxhyphen{} 保证“原始性”
:return:返回加载好的原始数据

\end{fulllineitems}

\index{save\_to\_mysql()(在 data\_loader 模块中)@\spxentry{save\_to\_mysql()(在 data\_loader 模块中)}}

\begin{fulllineitems}
\phantomsection\label{\detokenize{api/data_loader:data_loader.save_to_mysql}}
\pysigstartsignatures
\pysiglinewithargsret
{\sphinxcode{\sphinxupquote{data\_loader.}}\sphinxbfcode{\sphinxupquote{save\_to\_mysql}}}
{\sphinxparam{\DUrole{n}{df}\DUrole{p}{:}\DUrole{w}{ }\DUrole{n}{DataFrame}}\sphinxparamcomma \sphinxparam{\DUrole{n}{table\_name}\DUrole{p}{:}\DUrole{w}{ }\DUrole{n}{str}}\sphinxparamcomma \sphinxparam{\DUrole{n}{engine}}}
{}
\pysigstopsignatures
\sphinxAtStartPar
load\_raw\_data已经清洗过列名,去除空格,可以正常存入
先使用if下面的代码导入数据,然后使用if判断数据没问题
:param df:原始数据
:param table\_name:提前设定好的表名
:param engine:数据库

\end{fulllineitems}

\index{table\_has\_data()(在 data\_loader 模块中)@\spxentry{table\_has\_data()(在 data\_loader 模块中)}}

\begin{fulllineitems}
\phantomsection\label{\detokenize{api/data_loader:data_loader.table_has_data}}
\pysigstartsignatures
\pysiglinewithargsret
{\sphinxcode{\sphinxupquote{data\_loader.}}\sphinxbfcode{\sphinxupquote{table\_has\_data}}}
{\sphinxparam{\DUrole{n}{engine}}\sphinxparamcomma \sphinxparam{\DUrole{n}{table\_name}\DUrole{p}{:}\DUrole{w}{ }\DUrole{n}{str}}}
{{ $\rightarrow$ bool}}
\pysigstopsignatures\begin{quote}\begin{description}
\sphinxlineitem{参数}\begin{itemize}
\item {} 
\sphinxAtStartPar
\sphinxstyleliteralstrong{\sphinxupquote{engine}}

\item {} 
\sphinxAtStartPar
\sphinxstyleliteralstrong{\sphinxupquote{table\_name}}

\end{itemize}

\sphinxlineitem{返回}
\sphinxAtStartPar


\end{description}\end{quote}

\end{fulllineitems}


\sphinxstepscope


\section{data\_clean}
\label{\detokenize{api/data_clean:module-data_clean}}\label{\detokenize{api/data_clean:data-clean}}\label{\detokenize{api/data_clean::doc}}\index{module@\spxentry{module}!data\_clean@\spxentry{data\_clean}}\index{data\_clean@\spxentry{data\_clean}!module@\spxentry{module}}\subsubsection*{Functions}


\begin{savenotes}
\sphinxatlongtablestart
\sphinxthistablewithglobalstyle
\sphinxthistablewithnovlinesstyle
\makeatletter
  \LTleft \@totalleftmargin plus1fill
  \LTright\dimexpr\columnwidth-\@totalleftmargin-\linewidth\relax plus1fill
\makeatother
\begin{longtable}{\X{1}{2}\X{1}{2}}
\sphinxtoprule
\endfirsthead

\multicolumn{2}{c}{\sphinxnorowcolor
    \makebox[0pt]{\sphinxtablecontinued{\tablename\ \thetable{} \textendash{} 接上页}}%
}\\
\sphinxtoprule
\endhead

\sphinxbottomrule
\multicolumn{2}{r}{\sphinxnorowcolor
    \makebox[0pt][r]{\sphinxtablecontinued{续下页}}%
}\\
\endfoot

\endlastfoot
\sphinxtableatstartofbodyhook
\begin{varwidth}[t]{\sphinxcolwidth{1}{2}}
\sphinxAtStartPar
{\hyperref[\detokenize{api/data_clean:data_clean.data_clean}]{\sphinxcrossref{\sphinxcode{\sphinxupquote{data\_clean}}}}}(df, numeric\_cols, categorical\_cols)
\sphinxbeforeendvarwidth
\end{varwidth}%
&\begin{varwidth}[t]{\sphinxcolwidth{1}{2}}
\sphinxAtStartPar
数据清洗主函数 :return:
\sphinxbeforeendvarwidth
\end{varwidth}%
\\
\sphinxhline\begin{varwidth}[t]{\sphinxcolwidth{1}{2}}
\sphinxAtStartPar
{\hyperref[\detokenize{api/data_clean:data_clean.handle_categorical_missing}]{\sphinxcrossref{\sphinxcode{\sphinxupquote{handle\_categorical\_missing}}}}}(df, categorical\_cols)
\sphinxbeforeendvarwidth
\end{varwidth}%
&\begin{varwidth}[t]{\sphinxcolwidth{1}{2}}
\sphinxAtStartPar
类别型特征缺失值处理:填充为 'Unknown' :param df:原始数据集 :param categorical\_cols:类别型列 :param fill\_value:填充值,默认为 'Unknown' :return:填充后的数据
\sphinxbeforeendvarwidth
\end{varwidth}%
\\
\sphinxhline\begin{varwidth}[t]{\sphinxcolwidth{1}{2}}
\sphinxAtStartPar
{\hyperref[\detokenize{api/data_clean:data_clean.handle_missing_values}]{\sphinxcrossref{\sphinxcode{\sphinxupquote{handle\_missing\_values}}}}}(df, numeric\_cols{[}, ...{]})
\sphinxbeforeendvarwidth
\end{varwidth}%
&\begin{varwidth}[t]{\sphinxcolwidth{1}{2}}
\sphinxAtStartPar
数值型特征缺失值处理: 1.
\sphinxbeforeendvarwidth
\end{varwidth}%
\\
\sphinxhline\begin{varwidth}[t]{\sphinxcolwidth{1}{2}}
\sphinxAtStartPar
{\hyperref[\detokenize{api/data_clean:data_clean.mark_abnormal_values}]{\sphinxcrossref{\sphinxcode{\sphinxupquote{mark\_abnormal\_values}}}}}(df)
\sphinxbeforeendvarwidth
\end{varwidth}%
&\begin{varwidth}[t]{\sphinxcolwidth{1}{2}}
\sphinxAtStartPar
异常值标记(不修改原始数值) 当前仅对 age 进行简单规则标记 \sphinxhyphen{} age <= 0 或 age > 100 视为异常 :param df: DataFrame :return: 添加异常标记列后的 DataFrame数据集
\sphinxbeforeendvarwidth
\end{varwidth}%
\\
\sphinxhline\begin{varwidth}[t]{\sphinxcolwidth{1}{2}}
\sphinxAtStartPar
{\hyperref[\detokenize{api/data_clean:data_clean.remove_duplicates}]{\sphinxcrossref{\sphinxcode{\sphinxupquote{remove\_duplicates}}}}}(df)
\sphinxbeforeendvarwidth
\end{varwidth}%
&\begin{varwidth}[t]{\sphinxcolwidth{1}{2}}
\sphinxAtStartPar
重复值处理 \sphinxhyphen{} 统计重复行数量 \sphinxhyphen{} 删除完全重复的行(保留第一条) :param df: DataFrame :return: 去重后的 DataFrame数据集
\sphinxbeforeendvarwidth
\end{varwidth}%
\\
\sphinxhline\begin{varwidth}[t]{\sphinxcolwidth{1}{2}}
\sphinxAtStartPar
{\hyperref[\detokenize{api/data_clean:data_clean.save_clean_data}]{\sphinxcrossref{\sphinxcode{\sphinxupquote{save\_clean\_data}}}}}(df, path)
\sphinxbeforeendvarwidth
\end{varwidth}%
&\begin{varwidth}[t]{\sphinxcolwidth{1}{2}}
\sphinxAtStartPar
保存清洗后的数据 :param df: DataFrame :param path: 保存路径
\sphinxbeforeendvarwidth
\end{varwidth}%
\\
\sphinxbottomrule
\end{longtable}
\sphinxtableafterendhook
\sphinxatlongtableend
\end{savenotes}
\index{data\_clean()(在 data\_clean 模块中)@\spxentry{data\_clean()(在 data\_clean 模块中)}}

\begin{fulllineitems}
\phantomsection\label{\detokenize{api/data_clean:data_clean.data_clean}}
\pysigstartsignatures
\pysiglinewithargsret
{\sphinxcode{\sphinxupquote{data\_clean.}}\sphinxbfcode{\sphinxupquote{data\_clean}}}
{\sphinxparam{\DUrole{n}{df}\DUrole{p}{:}\DUrole{w}{ }\DUrole{n}{DataFrame}}\sphinxparamcomma \sphinxparam{\DUrole{n}{numeric\_cols}\DUrole{p}{:}\DUrole{w}{ }\DUrole{n}{list}}\sphinxparamcomma \sphinxparam{\DUrole{n}{categorical\_cols}\DUrole{p}{:}\DUrole{w}{ }\DUrole{n}{list}}}
{{ $\rightarrow$ DataFrame}}
\pysigstopsignatures
\sphinxAtStartPar
数据清洗主函数
:return:

\end{fulllineitems}

\index{handle\_categorical\_missing()(在 data\_clean 模块中)@\spxentry{handle\_categorical\_missing()(在 data\_clean 模块中)}}

\begin{fulllineitems}
\phantomsection\label{\detokenize{api/data_clean:data_clean.handle_categorical_missing}}
\pysigstartsignatures
\pysiglinewithargsret
{\sphinxcode{\sphinxupquote{data\_clean.}}\sphinxbfcode{\sphinxupquote{handle\_categorical\_missing}}}
{\sphinxparam{\DUrole{n}{df}\DUrole{p}{:}\DUrole{w}{ }\DUrole{n}{DataFrame}}\sphinxparamcomma \sphinxparam{\DUrole{n}{categorical\_cols}\DUrole{p}{:}\DUrole{w}{ }\DUrole{n}{list}}\sphinxparamcomma \sphinxparam{\DUrole{n}{fill\_value}\DUrole{p}{:}\DUrole{w}{ }\DUrole{n}{str}\DUrole{w}{ }\DUrole{o}{=}\DUrole{w}{ }\DUrole{default_value}{'Unknown'}}}
{{ $\rightarrow$ DataFrame}}
\pysigstopsignatures
\sphinxAtStartPar
类别型特征缺失值处理:填充为 'Unknown'
:param df:原始数据集
:param categorical\_cols:类别型列
:param fill\_value:填充值,默认为 'Unknown'
:return:填充后的数据

\end{fulllineitems}

\index{handle\_missing\_values()(在 data\_clean 模块中)@\spxentry{handle\_missing\_values()(在 data\_clean 模块中)}}

\begin{fulllineitems}
\phantomsection\label{\detokenize{api/data_clean:data_clean.handle_missing_values}}
\pysigstartsignatures
\pysiglinewithargsret
{\sphinxcode{\sphinxupquote{data\_clean.}}\sphinxbfcode{\sphinxupquote{handle\_missing\_values}}}
{\sphinxparam{\DUrole{n}{df}\DUrole{p}{:}\DUrole{w}{ }\DUrole{n}{DataFrame}}\sphinxparamcomma \sphinxparam{\DUrole{n}{numeric\_cols}\DUrole{p}{:}\DUrole{w}{ }\DUrole{n}{list}}\sphinxparamcomma \sphinxparam{\DUrole{n}{fill\_value}\DUrole{p}{:}\DUrole{w}{ }\DUrole{n}{int}\DUrole{w}{ }\DUrole{o}{=}\DUrole{w}{ }\DUrole{default_value}{\sphinxhyphen{}1}}}
{{ $\rightarrow$ DataFrame}}
\pysigstopsignatures
\sphinxAtStartPar
数值型特征缺失值处理:
1. 使用指定值(默认 \sphinxhyphen{}1)填充缺失
2. 为每个数值特征创建缺失指示变量
:param df:原始 DataFrame数据集
:param numeric\_cols:数值列
:return: 填充后的数据

\end{fulllineitems}

\index{mark\_abnormal\_values()(在 data\_clean 模块中)@\spxentry{mark\_abnormal\_values()(在 data\_clean 模块中)}}

\begin{fulllineitems}
\phantomsection\label{\detokenize{api/data_clean:data_clean.mark_abnormal_values}}
\pysigstartsignatures
\pysiglinewithargsret
{\sphinxcode{\sphinxupquote{data\_clean.}}\sphinxbfcode{\sphinxupquote{mark\_abnormal\_values}}}
{\sphinxparam{\DUrole{n}{df}\DUrole{p}{:}\DUrole{w}{ }\DUrole{n}{DataFrame}}}
{{ $\rightarrow$ DataFrame}}
\pysigstopsignatures
\sphinxAtStartPar
异常值标记(不修改原始数值)
当前仅对 age 进行简单规则标记
\sphinxhyphen{} age <= 0 或 age > 100 视为异常
:param df: DataFrame
:return: 添加异常标记列后的 DataFrame数据集

\end{fulllineitems}

\index{remove\_duplicates()(在 data\_clean 模块中)@\spxentry{remove\_duplicates()(在 data\_clean 模块中)}}

\begin{fulllineitems}
\phantomsection\label{\detokenize{api/data_clean:data_clean.remove_duplicates}}
\pysigstartsignatures
\pysiglinewithargsret
{\sphinxcode{\sphinxupquote{data\_clean.}}\sphinxbfcode{\sphinxupquote{remove\_duplicates}}}
{\sphinxparam{\DUrole{n}{df}\DUrole{p}{:}\DUrole{w}{ }\DUrole{n}{DataFrame}}}
{{ $\rightarrow$ DataFrame}}
\pysigstopsignatures
\sphinxAtStartPar
重复值处理
\sphinxhyphen{} 统计重复行数量
\sphinxhyphen{} 删除完全重复的行(保留第一条)
:param df: DataFrame
:return: 去重后的 DataFrame数据集

\end{fulllineitems}

\index{save\_clean\_data()(在 data\_clean 模块中)@\spxentry{save\_clean\_data()(在 data\_clean 模块中)}}

\begin{fulllineitems}
\phantomsection\label{\detokenize{api/data_clean:data_clean.save_clean_data}}
\pysigstartsignatures
\pysiglinewithargsret
{\sphinxcode{\sphinxupquote{data\_clean.}}\sphinxbfcode{\sphinxupquote{save\_clean\_data}}}
{\sphinxparam{\DUrole{n}{df}\DUrole{p}{:}\DUrole{w}{ }\DUrole{n}{DataFrame}}\sphinxparamcomma \sphinxparam{\DUrole{n}{path}\DUrole{p}{:}\DUrole{w}{ }\DUrole{n}{str}}}
{{ $\rightarrow$ None}}
\pysigstopsignatures
\sphinxAtStartPar
保存清洗后的数据
:param df: DataFrame
:param path: 保存路径

\end{fulllineitems}


\sphinxstepscope


\section{data\_explore}
\label{\detokenize{api/data_explore:module-data_explore}}\label{\detokenize{api/data_explore:data-explore}}\label{\detokenize{api/data_explore::doc}}\index{module@\spxentry{module}!data\_explore@\spxentry{data\_explore}}\index{data\_explore@\spxentry{data\_explore}!module@\spxentry{module}}\subsubsection*{Functions}


\begin{savenotes}
\sphinxatlongtablestart
\sphinxthistablewithglobalstyle
\sphinxthistablewithnovlinesstyle
\makeatletter
  \LTleft \@totalleftmargin plus1fill
  \LTright\dimexpr\columnwidth-\@totalleftmargin-\linewidth\relax plus1fill
\makeatother
\begin{longtable}{\X{1}{2}\X{1}{2}}
\sphinxtoprule
\endfirsthead

\multicolumn{2}{c}{\sphinxnorowcolor
    \makebox[0pt]{\sphinxtablecontinued{\tablename\ \thetable{} \textendash{} 接上页}}%
}\\
\sphinxtoprule
\endhead

\sphinxbottomrule
\multicolumn{2}{r}{\sphinxnorowcolor
    \makebox[0pt][r]{\sphinxtablecontinued{续下页}}%
}\\
\endfoot

\endlastfoot
\sphinxtableatstartofbodyhook
\begin{varwidth}[t]{\sphinxcolwidth{1}{2}}
\sphinxAtStartPar
{\hyperref[\detokenize{api/data_explore:data_explore.analyze_feature_by_group}]{\sphinxcrossref{\sphinxcode{\sphinxupquote{analyze\_feature\_by\_group}}}}}(df, group\_col, ...)
\sphinxbeforeendvarwidth
\end{varwidth}%
&\begin{varwidth}[t]{\sphinxcolwidth{1}{2}}
\sphinxAtStartPar
分析某个特征在不同分组中的分布 :param df: pandas DataFrame,输入的数据集 :param group\_col:str,分组列名(如'lifecycle') :param feature\_col:str,要分析的特征列名 :param normalize:bool,是否计算比例而不是计数,默认为True(计算比例) :return:DataFrame,交叉表显示特征在不同分组中的分布
\sphinxbeforeendvarwidth
\end{varwidth}%
\\
\sphinxhline\begin{varwidth}[t]{\sphinxcolwidth{1}{2}}
\sphinxAtStartPar
{\hyperref[\detokenize{api/data_explore:data_explore.data_explore}]{\sphinxcrossref{\sphinxcode{\sphinxupquote{data\_explore}}}}}()
\sphinxbeforeendvarwidth
\end{varwidth}%
&\begin{varwidth}[t]{\sphinxcolwidth{1}{2}}
\sphinxAtStartPar
展示所有探索的数据 :return:
\sphinxbeforeendvarwidth
\end{varwidth}%
\\
\sphinxhline\begin{varwidth}[t]{\sphinxcolwidth{1}{2}}
\sphinxAtStartPar
{\hyperref[\detokenize{api/data_explore:data_explore.explore_categorical_features}]{\sphinxcrossref{\sphinxcode{\sphinxupquote{explore\_categorical\_features}}}}}(df{[}, top\_n{]})
\sphinxbeforeendvarwidth
\end{varwidth}%
&\begin{varwidth}[t]{\sphinxcolwidth{1}{2}}
\sphinxAtStartPar
分析类别型特征的分布情况 :param df:输入原始的数据集 :param top\_n:int,显示每个类别的前N个值,默认显示前10个 :return:字典,键为类别型列名,值为该列的值分布Series
\sphinxbeforeendvarwidth
\end{varwidth}%
\\
\sphinxhline\begin{varwidth}[t]{\sphinxcolwidth{1}{2}}
\sphinxAtStartPar
{\hyperref[\detokenize{api/data_explore:data_explore.explore_correlation}]{\sphinxcrossref{\sphinxcode{\sphinxupquote{explore\_correlation}}}}}(df{[}, method, threshold{]})
\sphinxbeforeendvarwidth
\end{varwidth}%
&\begin{varwidth}[t]{\sphinxcolwidth{1}{2}}
\sphinxAtStartPar
分析数值特征之间的相关性 :param df:DataFrame,原始数据 :param method: 相关系数计算方法 ('pearson', 'spearman', 'kendall'),默认 pearson :param threshold: 相关性阈值,用于筛选强相关特征对 :return:     corr\_matrix : 相关性矩阵     strong\_corr : 强相关特征对(DataFrame)
\sphinxbeforeendvarwidth
\end{varwidth}%
\\
\sphinxhline\begin{varwidth}[t]{\sphinxcolwidth{1}{2}}
\sphinxAtStartPar
{\hyperref[\detokenize{api/data_explore:data_explore.explore_missing_values}]{\sphinxcrossref{\sphinxcode{\sphinxupquote{explore\_missing\_values}}}}}(df)
\sphinxbeforeendvarwidth
\end{varwidth}%
&\begin{varwidth}[t]{\sphinxcolwidth{1}{2}}
\sphinxAtStartPar
统计各字段缺失率 :return:DataFrame,包含每个字段的缺失数量和缺失率,按缺失率降序排列
\sphinxbeforeendvarwidth
\end{varwidth}%
\\
\sphinxhline\begin{varwidth}[t]{\sphinxcolwidth{1}{2}}
\sphinxAtStartPar
{\hyperref[\detokenize{api/data_explore:data_explore.explore_numeric_features}]{\sphinxcrossref{\sphinxcode{\sphinxupquote{explore\_numeric\_features}}}}}(df)
\sphinxbeforeendvarwidth
\end{varwidth}%
&\begin{varwidth}[t]{\sphinxcolwidth{1}{2}}
\sphinxAtStartPar
分析数值型特征的描述性统计 :param df:输入原始的数据集 :return:DataFrame,包含各数值型特征的统计量(计数、均值、标准差、最小值、25\%、50\%、75\%、最大值)
\sphinxbeforeendvarwidth
\end{varwidth}%
\\
\sphinxhline\begin{varwidth}[t]{\sphinxcolwidth{1}{2}}
\sphinxAtStartPar
{\hyperref[\detokenize{api/data_explore:data_explore.split_columns_by_type}]{\sphinxcrossref{\sphinxcode{\sphinxupquote{split\_columns\_by\_type}}}}}(df)
\sphinxbeforeendvarwidth
\end{varwidth}%
&\begin{varwidth}[t]{\sphinxcolwidth{1}{2}}
\sphinxAtStartPar
自动划分数值列和类别列 :param df: DataFrame数据集 :return: 数值列和类别列
\sphinxbeforeendvarwidth
\end{varwidth}%
\\
\sphinxbottomrule
\end{longtable}
\sphinxtableafterendhook
\sphinxatlongtableend
\end{savenotes}
\index{analyze\_feature\_by\_group()(在 data\_explore 模块中)@\spxentry{analyze\_feature\_by\_group()(在 data\_explore 模块中)}}

\begin{fulllineitems}
\phantomsection\label{\detokenize{api/data_explore:data_explore.analyze_feature_by_group}}
\pysigstartsignatures
\pysiglinewithargsret
{\sphinxcode{\sphinxupquote{data\_explore.}}\sphinxbfcode{\sphinxupquote{analyze\_feature\_by\_group}}}
{\sphinxparam{\DUrole{n}{df}\DUrole{p}{:}\DUrole{w}{ }\DUrole{n}{DataFrame}}\sphinxparamcomma \sphinxparam{\DUrole{n}{group\_col}\DUrole{p}{:}\DUrole{w}{ }\DUrole{n}{str}}\sphinxparamcomma \sphinxparam{\DUrole{n}{feature\_col}\DUrole{p}{:}\DUrole{w}{ }\DUrole{n}{str}}\sphinxparamcomma \sphinxparam{\DUrole{n}{normalize}\DUrole{p}{:}\DUrole{w}{ }\DUrole{n}{bool}\DUrole{w}{ }\DUrole{o}{=}\DUrole{w}{ }\DUrole{default_value}{True}}}
{{ $\rightarrow$ DataFrame}}
\pysigstopsignatures
\sphinxAtStartPar
分析某个特征在不同分组中的分布
:param df: pandas DataFrame,输入的数据集
:param group\_col:str,分组列名(如'lifecycle')
:param feature\_col:str,要分析的特征列名
:param normalize:bool,是否计算比例而不是计数,默认为True(计算比例)
:return:DataFrame,交叉表显示特征在不同分组中的分布

\end{fulllineitems}

\index{data\_explore()(在 data\_explore 模块中)@\spxentry{data\_explore()(在 data\_explore 模块中)}}

\begin{fulllineitems}
\phantomsection\label{\detokenize{api/data_explore:data_explore.data_explore}}
\pysigstartsignatures
\pysiglinewithargsret
{\sphinxcode{\sphinxupquote{data\_explore.}}\sphinxbfcode{\sphinxupquote{data\_explore}}}
{}
{}
\pysigstopsignatures
\sphinxAtStartPar
展示所有探索的数据
:return:

\end{fulllineitems}

\index{explore\_categorical\_features()(在 data\_explore 模块中)@\spxentry{explore\_categorical\_features()(在 data\_explore 模块中)}}

\begin{fulllineitems}
\phantomsection\label{\detokenize{api/data_explore:data_explore.explore_categorical_features}}
\pysigstartsignatures
\pysiglinewithargsret
{\sphinxcode{\sphinxupquote{data\_explore.}}\sphinxbfcode{\sphinxupquote{explore\_categorical\_features}}}
{\sphinxparam{\DUrole{n}{df}\DUrole{p}{:}\DUrole{w}{ }\DUrole{n}{DataFrame}}\sphinxparamcomma \sphinxparam{\DUrole{n}{top\_n}\DUrole{p}{:}\DUrole{w}{ }\DUrole{n}{int}\DUrole{w}{ }\DUrole{o}{=}\DUrole{w}{ }\DUrole{default_value}{10}}}
{{ $\rightarrow$ dict}}
\pysigstopsignatures
\sphinxAtStartPar
分析类别型特征的分布情况
:param df:输入原始的数据集
:param top\_n:int,显示每个类别的前N个值,默认显示前10个
:return:字典,键为类别型列名,值为该列的值分布Series

\end{fulllineitems}

\index{explore\_correlation()(在 data\_explore 模块中)@\spxentry{explore\_correlation()(在 data\_explore 模块中)}}

\begin{fulllineitems}
\phantomsection\label{\detokenize{api/data_explore:data_explore.explore_correlation}}
\pysigstartsignatures
\pysiglinewithargsret
{\sphinxcode{\sphinxupquote{data\_explore.}}\sphinxbfcode{\sphinxupquote{explore\_correlation}}}
{\sphinxparam{\DUrole{n}{df}\DUrole{p}{:}\DUrole{w}{ }\DUrole{n}{DataFrame}}\sphinxparamcomma \sphinxparam{\DUrole{n}{method}\DUrole{p}{:}\DUrole{w}{ }\DUrole{n}{str}\DUrole{w}{ }\DUrole{o}{=}\DUrole{w}{ }\DUrole{default_value}{'pearson'}}\sphinxparamcomma \sphinxparam{\DUrole{n}{threshold}\DUrole{p}{:}\DUrole{w}{ }\DUrole{n}{float}\DUrole{w}{ }\DUrole{o}{=}\DUrole{w}{ }\DUrole{default_value}{0.7}}}
{{ $\rightarrow$ DataFrame}}
\pysigstopsignatures
\sphinxAtStartPar
分析数值特征之间的相关性
:param df:DataFrame,原始数据
:param method: 相关系数计算方法 ('pearson', 'spearman', 'kendall'),默认 pearson
:param threshold: 相关性阈值,用于筛选强相关特征对
:return:
\begin{quote}

\sphinxAtStartPar
corr\_matrix : 相关性矩阵
strong\_corr : 强相关特征对(DataFrame)
\end{quote}

\end{fulllineitems}

\index{explore\_missing\_values()(在 data\_explore 模块中)@\spxentry{explore\_missing\_values()(在 data\_explore 模块中)}}

\begin{fulllineitems}
\phantomsection\label{\detokenize{api/data_explore:data_explore.explore_missing_values}}
\pysigstartsignatures
\pysiglinewithargsret
{\sphinxcode{\sphinxupquote{data\_explore.}}\sphinxbfcode{\sphinxupquote{explore\_missing\_values}}}
{\sphinxparam{\DUrole{n}{df}\DUrole{p}{:}\DUrole{w}{ }\DUrole{n}{DataFrame}}}
{{ $\rightarrow$ DataFrame}}
\pysigstopsignatures
\sphinxAtStartPar
统计各字段缺失率
:return:DataFrame,包含每个字段的缺失数量和缺失率,按缺失率降序排列

\end{fulllineitems}

\index{explore\_numeric\_features()(在 data\_explore 模块中)@\spxentry{explore\_numeric\_features()(在 data\_explore 模块中)}}

\begin{fulllineitems}
\phantomsection\label{\detokenize{api/data_explore:data_explore.explore_numeric_features}}
\pysigstartsignatures
\pysiglinewithargsret
{\sphinxcode{\sphinxupquote{data\_explore.}}\sphinxbfcode{\sphinxupquote{explore\_numeric\_features}}}
{\sphinxparam{\DUrole{n}{df}\DUrole{p}{:}\DUrole{w}{ }\DUrole{n}{DataFrame}}}
{{ $\rightarrow$ DataFrame}}
\pysigstopsignatures
\sphinxAtStartPar
分析数值型特征的描述性统计
:param df:输入原始的数据集
:return:DataFrame,包含各数值型特征的统计量(计数、均值、标准差、最小值、25\%、50\%、75\%、最大值)

\end{fulllineitems}

\index{split\_columns\_by\_type()(在 data\_explore 模块中)@\spxentry{split\_columns\_by\_type()(在 data\_explore 模块中)}}

\begin{fulllineitems}
\phantomsection\label{\detokenize{api/data_explore:data_explore.split_columns_by_type}}
\pysigstartsignatures
\pysiglinewithargsret
{\sphinxcode{\sphinxupquote{data\_explore.}}\sphinxbfcode{\sphinxupquote{split\_columns\_by\_type}}}
{\sphinxparam{\DUrole{n}{df}\DUrole{p}{:}\DUrole{w}{ }\DUrole{n}{DataFrame}}}
{}
\pysigstopsignatures
\sphinxAtStartPar
自动划分数值列和类别列
:param df: DataFrame数据集
:return: 数值列和类别列

\end{fulllineitems}


\sphinxstepscope


\section{model\_train}
\label{\detokenize{api/model_train:module-model_train}}\label{\detokenize{api/model_train:model-train}}\label{\detokenize{api/model_train::doc}}\index{module@\spxentry{module}!model\_train@\spxentry{model\_train}}\index{model\_train@\spxentry{model\_train}!module@\spxentry{module}}\subsubsection*{Functions}


\begin{savenotes}
\sphinxatlongtablestart
\sphinxthistablewithglobalstyle
\sphinxthistablewithnovlinesstyle
\makeatletter
  \LTleft \@totalleftmargin plus1fill
  \LTright\dimexpr\columnwidth-\@totalleftmargin-\linewidth\relax plus1fill
\makeatother
\begin{longtable}{\X{1}{2}\X{1}{2}}
\sphinxtoprule
\endfirsthead

\multicolumn{2}{c}{\sphinxnorowcolor
    \makebox[0pt]{\sphinxtablecontinued{\tablename\ \thetable{} \textendash{} 接上页}}%
}\\
\sphinxtoprule
\endhead

\sphinxbottomrule
\multicolumn{2}{r}{\sphinxnorowcolor
    \makebox[0pt][r]{\sphinxtablecontinued{续下页}}%
}\\
\endfoot

\endlastfoot
\sphinxtableatstartofbodyhook
\begin{varwidth}[t]{\sphinxcolwidth{1}{2}}
\sphinxAtStartPar
{\hyperref[\detokenize{api/model_train:model_train.train_model}]{\sphinxcrossref{\sphinxcode{\sphinxupquote{train\_model}}}}}(df, target\_col{[}, test\_size, ...{]})
\sphinxbeforeendvarwidth
\end{varwidth}%
&\begin{varwidth}[t]{\sphinxcolwidth{1}{2}}
\sphinxAtStartPar
模型训练模块 :param df: 特征工程后的数据 :param target\_col: 目标列 :return: 训练好的模型 + 测试集
\sphinxbeforeendvarwidth
\end{varwidth}%
\\
\sphinxbottomrule
\end{longtable}
\sphinxtableafterendhook
\sphinxatlongtableend
\end{savenotes}
\index{train\_model()(在 model\_train 模块中)@\spxentry{train\_model()(在 model\_train 模块中)}}

\begin{fulllineitems}
\phantomsection\label{\detokenize{api/model_train:model_train.train_model}}
\pysigstartsignatures
\pysiglinewithargsret
{\sphinxcode{\sphinxupquote{model\_train.}}\sphinxbfcode{\sphinxupquote{train\_model}}}
{\sphinxparam{\DUrole{n}{df}\DUrole{p}{:}\DUrole{w}{ }\DUrole{n}{DataFrame}}\sphinxparamcomma \sphinxparam{\DUrole{n}{target\_col}\DUrole{p}{:}\DUrole{w}{ }\DUrole{n}{str}}\sphinxparamcomma \sphinxparam{\DUrole{n}{test\_size}\DUrole{p}{:}\DUrole{w}{ }\DUrole{n}{float}\DUrole{w}{ }\DUrole{o}{=}\DUrole{w}{ }\DUrole{default_value}{0.2}}\sphinxparamcomma \sphinxparam{\DUrole{n}{random\_state}\DUrole{p}{:}\DUrole{w}{ }\DUrole{n}{int}\DUrole{w}{ }\DUrole{o}{=}\DUrole{w}{ }\DUrole{default_value}{123}}\sphinxparamcomma \sphinxparam{\DUrole{n}{model\_type}\DUrole{p}{:}\DUrole{w}{ }\DUrole{n}{str}\DUrole{w}{ }\DUrole{o}{=}\DUrole{w}{ }\DUrole{default_value}{'rf'}}}
{}
\pysigstopsignatures
\sphinxAtStartPar
模型训练模块
:param df: 特征工程后的数据
:param target\_col: 目标列
:return: 训练好的模型 + 测试集

\end{fulllineitems}


\sphinxstepscope


\section{model\_evaluate}
\label{\detokenize{api/model_evaluate:module-model_evaluate}}\label{\detokenize{api/model_evaluate:model-evaluate}}\label{\detokenize{api/model_evaluate::doc}}\index{module@\spxentry{module}!model\_evaluate@\spxentry{model\_evaluate}}\index{model\_evaluate@\spxentry{model\_evaluate}!module@\spxentry{module}}\subsubsection*{Functions}


\begin{savenotes}
\sphinxatlongtablestart
\sphinxthistablewithglobalstyle
\sphinxthistablewithnovlinesstyle
\makeatletter
  \LTleft \@totalleftmargin plus1fill
  \LTright\dimexpr\columnwidth-\@totalleftmargin-\linewidth\relax plus1fill
\makeatother
\begin{longtable}{\X{1}{2}\X{1}{2}}
\sphinxtoprule
\endfirsthead

\multicolumn{2}{c}{\sphinxnorowcolor
    \makebox[0pt]{\sphinxtablecontinued{\tablename\ \thetable{} \textendash{} 接上页}}%
}\\
\sphinxtoprule
\endhead

\sphinxbottomrule
\multicolumn{2}{r}{\sphinxnorowcolor
    \makebox[0pt][r]{\sphinxtablecontinued{续下页}}%
}\\
\endfoot

\endlastfoot
\sphinxtableatstartofbodyhook
\begin{varwidth}[t]{\sphinxcolwidth{1}{2}}
\sphinxAtStartPar
{\hyperref[\detokenize{api/model_evaluate:model_evaluate.evaluate_model}]{\sphinxcrossref{\sphinxcode{\sphinxupquote{evaluate\_model}}}}}(model, x\_test, y\_test)
\sphinxbeforeendvarwidth
\end{varwidth}%
&\begin{varwidth}[t]{\sphinxcolwidth{1}{2}}
\sphinxAtStartPar
模型评估函数
\sphinxbeforeendvarwidth
\end{varwidth}%
\\
\sphinxbottomrule
\end{longtable}
\sphinxtableafterendhook
\sphinxatlongtableend
\end{savenotes}
\index{evaluate\_model()(在 model\_evaluate 模块中)@\spxentry{evaluate\_model()(在 model\_evaluate 模块中)}}

\begin{fulllineitems}
\phantomsection\label{\detokenize{api/model_evaluate:model_evaluate.evaluate_model}}
\pysigstartsignatures
\pysiglinewithargsret
{\sphinxcode{\sphinxupquote{model\_evaluate.}}\sphinxbfcode{\sphinxupquote{evaluate\_model}}}
{\sphinxparam{\DUrole{n}{model}}\sphinxparamcomma \sphinxparam{\DUrole{n}{x\_test}}\sphinxparamcomma \sphinxparam{\DUrole{n}{y\_test}}}
{}
\pysigstopsignatures
\sphinxAtStartPar
模型评估函数

\end{fulllineitems}


\sphinxstepscope


\section{config}
\label{\detokenize{api/config:module-config}}\label{\detokenize{api/config:config}}\label{\detokenize{api/config::doc}}\index{module@\spxentry{module}!config@\spxentry{config}}\index{config@\spxentry{config}!module@\spxentry{module}}

\chapter{Indices and tables}
\label{\detokenize{index:indices-and-tables}}\begin{itemize}
\item {} 
\sphinxAtStartPar
\DUrole{xref}{\DUrole{std}{\DUrole{std-ref}{genindex}}}

\item {} 
\sphinxAtStartPar
\DUrole{xref}{\DUrole{std}{\DUrole{std-ref}{modindex}}}

\item {} 
\sphinxAtStartPar
\DUrole{xref}{\DUrole{std}{\DUrole{std-ref}{search}}}

\end{itemize}


\chapter{项目介绍}
\label{\detokenize{index:id3}}
\sphinxAtStartPar
EcommerceSalesAnalysis 是一个电商平台销售数据分析项目。该项目旨在通过数据科学的方法对电商平台的销售数据进行深入分析,提供有价值的业务洞察。


\section{功能特性}
\label{\detokenize{index:id4}}\begin{itemize}
\item {} 
\sphinxAtStartPar
\sphinxstylestrong{数据加载}: 支持多种格式的销售数据加载

\item {} 
\sphinxAtStartPar
\sphinxstylestrong{数据清洗}: 自动识别和处理数据中的异常值、缺失值

\item {} 
\sphinxAtStartPar
\sphinxstylestrong{数据探索}: 提供丰富的统计分析功能

\item {} 
\sphinxAtStartPar
\sphinxstylestrong{数据可视化}: 生成直观的图表展示数据特征

\item {} 
\sphinxAtStartPar
\sphinxstylestrong{特征工程}: 提取有用的特征用于模型训练

\item {} 
\sphinxAtStartPar
\sphinxstylestrong{模型训练}: 使用机器学习算法预测销售趋势

\item {} 
\sphinxAtStartPar
\sphinxstylestrong{模型评估}: 评估模型性能并提供可视化结果

\end{itemize}


\section{技术栈}
\label{\detokenize{index:id5}}\begin{itemize}
\item {} 
\sphinxAtStartPar
Python 3.x

\item {} 
\sphinxAtStartPar
Pandas \sphinxhyphen{} 数据处理

\item {} 
\sphinxAtStartPar
NumPy \sphinxhyphen{} 数值计算

\item {} 
\sphinxAtStartPar
Matplotlib/Seaborn \sphinxhyphen{} 数据可视化

\item {} 
\sphinxAtStartPar
Scikit\sphinxhyphen{}learn \sphinxhyphen{} 机器学习

\item {} 
\sphinxAtStartPar
Sphinx \sphinxhyphen{} 文档生成

\end{itemize}


\renewcommand{\indexname}{Python 模块索引}
\begin{sphinxtheindex}
\let\bigletter\sphinxstyleindexlettergroup
\bigletter{c}
\item\relax\sphinxstyleindexentry{config}\sphinxstyleindexpageref{api/config:\detokenize{module-config}}
\indexspace
\bigletter{d}
\item\relax\sphinxstyleindexentry{data\_clean}\sphinxstyleindexpageref{api/data_clean:\detokenize{module-data_clean}}
\item\relax\sphinxstyleindexentry{data\_explore}\sphinxstyleindexpageref{api/data_explore:\detokenize{module-data_explore}}
\item\relax\sphinxstyleindexentry{data\_loader}\sphinxstyleindexpageref{api/data_loader:\detokenize{module-data_loader}}
\indexspace
\bigletter{m}
\item\relax\sphinxstyleindexentry{model\_evaluate}\sphinxstyleindexpageref{api/model_evaluate:\detokenize{module-model_evaluate}}
\item\relax\sphinxstyleindexentry{model\_train}\sphinxstyleindexpageref{api/model_train:\detokenize{module-model_train}}
\end{sphinxtheindex}

\renewcommand{\indexname}{索引}
\printindex
\end{document}